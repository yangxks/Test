\documentclass[a4paper]{article}
\usepackage{fontspec}
\usepackage{graphicx}
\usepackage{geometry}
\usepackage{titlesec}
%\setmainfont{Adobe Kaiti Std}
\setmainfont{STXihei}
\XeTeXlinebreaklocale "zh"
\XeTeXlinebreakskip = 0pt plus 1pt minus 0.1pt
\geometry{left=2cm,right=2cm,top=2.2cm,bottom=3cm}

%设置缩进
\setlength{\parindent}{4em}
%设置行距
\linespread{1.6}
%设置段间距
\setlength{\parskip}{3\baselineskip}

\newcommand{\zihao}{\fontsize{21pt}{\baselineskip}\selectfont}
\newcommand{\xiaochuhao}{\fontsize{36pt}{\baselineskip}\selectfont}
\newcommand{\xiaoerhao}{\fontsize{18pt}{\baselineskip}\selectfont}

\titleformat{\section}{\centering\large\bfseries}{\thesection}{1em}{}

\title{送东阳马生序}
\author{\href{mailto:yangxks@gmail.com}{宋 濂}}
\usepackage[colorlinks,bookmarks=false,pdfstartview=FitH,pdftitle=送东阳马生序,pdfauthor=leafduo]{hyperref}

\begin{document}
\zihao
\begin{center}
 {\bfseries\xiaochuhao 送东阳马生序}\\[1cm]
 \begin{flushright}
{\scshape\xiaoerhao 明\ 宋 濂}
 \end{flushright}
\end{center}


   余幼时即嗜学。家贫,无从致书以观,每假借于藏书之家,手自笔录,计日以还。天大寒,砚冰坚,手指不可屈伸,弗之怠。
录毕,走送之,不敢稍逾约。以是人多以书假余,余因得遍观群书。既加冠,益慕圣贤之道。又患无硕师名人与游,尝趋百里外。
从乡之先达执经叩问,先达德隆尊,门人弟子填其室,未尝稍降辞色。余立侍左右,援疑质理,俯身倾耳以请;或遇其叱咄,色
愈恭,礼愈至,不敢出一言以复;俟其欣悦,则又请焉。故余虽愚,卒获有所闻。


   当余之从师也,负箧曳屣,行深山巨谷中。穷冬烈风,大雪深数尺,足肤皲裂而不知。至舍,四支僵劲不能动,媵人持汤沃灌,
以衾拥覆,久而乃和。寓逆旅,主人日再食,无鲜肥滋味之享。同舍生皆被绮绣,戴朱缨宝饰之帽,腰白玉之环,左佩刀,右备容
臭,烨然若神人;余则缊袍敝衣处其间,略无慕艳意。以中有足乐者,不知口体之奉不若人也。盖余之勤且艰若此。今虽耄老,未
有所成,犹幸预君子之列,而承天子之宠光,缀公卿之后,日待坐,备顾问,四海亦谬称其氏名,况才之过于余者乎?


   今诸生学于太学,县官日有廪稍之供,父母岁有裘葛之遗,无冻馁之患矣;坐大厦之下而诵《诗》《书》,无奔走之劳矣;有
司业、博士为之师,未有问而不告,求而不得者也;凡所宜有之书皆集于此,不必若余之手录,假诸人而后见也。其业有不精,德
有不成者,非天质之卑,则心不若余之专耳,岂他人之过哉?

 
   东阳马生君则在太学已二年,流辈甚称其贤。余朝京师,生以乡人子谒余。撰长书以为贽,辞甚畅达。与之论辨,言和而色夷。
自谓少时用心于学甚劳。是可谓善学者矣。其将归见其亲也,余故道为学之难以告之。

\end{document}
